\section{Experiments}

%We have provided a path algorithm for our statistic, $\hat C$, which was motivated from first principles.
We will now conclude with an empirical study of the effectiveness of Algorithm \ref{alg:flowpath} on some simulations.
Notice that a specific max flow algorithm was not prescribed when computing the gradient flow.
In our implementation, we use the Edmonds-Karp algorithm, in which residual flows are found using breadth-first search.
More recently, a dual-decomposition algorithm has been developed in order to parallelize the computation of the MAP estimator for binary MRFs \cite{strandmark2010parallel,sontag2011introduction}.

\begin{figure*}
\centering{
\includegraphics[width=4cm]{rect_n=15_true.eps}\hspace{1cm}
\includegraphics[width=4cm]{rect_n=15_noisy.eps}\hspace{1cm}
\includegraphics[width=4cm]{rect_n=15_hat.eps}}
\caption{The undirected $25 \times 25$ lattice graph was used with a $4 \times 4$ rectangular region $C^\star$ (left).  The signal size used was $\mu = 2.35$ with $\sigma = 1$ ($\yb$ is depicted in the middle).  The best reconstruction is displayed (right).}
\label{fig1}
%\vspace{-.25cm}
\end{figure*}


\begin{figure*}
\centering{
\includegraphics[width=4cm]{dir_n=15_true.eps}\hspace{1cm}
\includegraphics[width=4cm]{dir_n=15_noisy.eps}\hspace{1cm}
\includegraphics[width=4cm]{dir_n=15_hat.eps}}
\caption{A directed graph (depicted by the arrows) was constructed with edges between all $8$ neighboring blocks in which more weight is on edges that descend to the right.  A region was chosen that has a low cut size (left).  The signal size is $\mu = 1.75$ and $\sigma = 1$ ($\yb$ is depicted in the middle).  The best reconstruction is displayed (right).}
\label{fig2}
\vspace{-.25cm}
\end{figure*}

\begin{figure*}
\centering{
\includegraphics[width=5.5cm]{rect_n=15_MSE_path.eps}
\includegraphics[width=5.5cm]{dir_n=15_MSE_path.eps}}
\caption{MSE as a function of regularization parameter $\nu$.  The lattice graph example is left and the directed graph example is right.}
\label{fig3}
\vspace{-.25cm}
\end{figure*}

We construct an undirected, unweighted lattice graph by identifying each vertex with a square in a $15 \times 15$ grid, and adjoining vertices that share an side of the square with weight $1$.
A $4 \times 4$ rectangle was constructed to be $C^\star$ and the signal size $\mu = 2.35$ with noise level $\sigma = 1$. (Figure \ref{fig1} depicts the cluster, noisy observations and reconstruction.)
The smallest MSE (Hamming distance between $C^\star$ and $\hat C$) in the regularization path was $2$. (The MSE throughout the regularization path is given in Figure \ref{fig3} (left)).

We form a weighted directed graph by associating the vertices to squares in a grid ($\{(i,j) : i, j \in [15]\}$) such that $(i_1,j_1), (i_2,j_2)$ have an edge between them if $|i_1 - i_2| \le 1$ and $|j_1 - j_2| \le 1$.  
Moreover, the edge weight is equal to $e^{(i_2 - i_1 + j_2 - i_1)/3}$ so that the weight is larger in the direction of the arrows of Figure \ref{fig2}.
$C^\star$ contains $16$ vertices depicted in Figure \ref{fig2} (left), the signal size is $\mu = 1.75$ and $\sigma = 1$.
Again the MSE is given in Figure \ref{fig3} (right).
It has been demonstrated that the RGS can successfully reconstruct the true cluster of activation $C^\star$.
This can also be done efficiently with the Flow Path algorithm, which gives us the entire regularization path for the RGS.
The algorithm above is computationally slower than the spanning tree wavelet \cite{sharpnack2012detecting}, but it experimentally dominates it and other pre-existing methods \cite{sharpnack2013near-optimal}.
%The graph scan statistic and the regularized graph scan are novel because they are able to scan over a combinatorial class of clusters defined using graph cuts.
