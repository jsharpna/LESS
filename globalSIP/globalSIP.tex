
%% bare_conf.tex
%% V1.3
%% 2007/01/11
%% by Michael Shell
%% See:
%% http://www.michaelshell.org/
%% for current contact information.
%%
%% This is a skeleton file demonstrating the use of IEEEtran.cls
%% (requires IEEEtran.cls version 1.7 or later) with an IEEE conference paper.
%%
%% Support sites:
%% http://www.michaelshell.org/tex/ieeetran/
%% http://www.ctan.org/tex-archive/macros/latex/contrib/IEEEtran/
%% and
%% http://www.ieee.org/

%%*************************************************************************
%% Legal Notice:
%% This code is offered as-is without any warranty either expressed or
%% implied; without even the implied warranty of MERCHANTABILITY or
%% FITNESS FOR A PARTICULAR PURPOSE!
%% User assumes all risk.
%% In no event shall IEEE or any contributor to this code be liable for
%% any damages or losses, including, but not limited to, incidental,
%% consequential, or any other damages, resulting from the use or misuse
%% of any information contained here.
%%
%% All comments are the opinions of their respective authors and are not
%% necessarily endorsed by the IEEE.
%%
%% This work is distributed under the LaTeX Project Public License (LPPL)
%% ( http://www.latex-project.org/ ) version 1.3, and may be freely used,
%% distributed and modified. A copy of the LPPL, version 1.3, is included
%% in the base LaTeX documentation of all distributions of LaTeX released
%% 2003/12/01 or later.
%% Retain all contribution notices and credits.
%% ** Modified files should be clearly indicated as such, including  **
%% ** renaming them and changing author support contact information. **
%%
%% File list of work: IEEEtran.cls, IEEEtran_HOWTO.pdf, bare_adv.tex,
%%                    bare_conf.tex, bare_jrnl.tex, bare_jrnl_compsoc.tex
%%*************************************************************************

% *** Authors should verify (and, if needed, correct) their LaTeX system  ***
% *** with the testflow diagnostic prior to trusting their LaTeX platform ***
% *** with production work. IEEE's font choices can trigger bugs that do  ***
% *** not appear when using other class files.                            ***
% The testflow support page is at:
% http://www.michaelshell.org/tex/testflow/



% Note that the a4paper option is mainly intended so that authors in
% countries using A4 can easily print to A4 and see how their papers will
% look in print - the typesetting of the document will not typically be
% affected with changes in paper size (but the bottom and side margins will).
% Use the testflow package mentioned above to verify correct handling of
% both paper sizes by the user's LaTeX system.
%
% Also note that the "draftcls" or "draftclsnofoot", not "draft", option
% should be used if it is desired that the figures are to be displayed in
% draft mode.
%
\documentclass[conference, 10pt]{IEEEtran}
% Add the compsoc option for Computer Society conferences.
%
% If IEEEtran.cls has not been installed into the LaTeX system files,
% manually specify the path to it like:
% \documentclass[conference]{../sty/IEEEtran}





% Some very useful LaTeX packages include:
% (uncomment the ones you want to load)


% *** MISC UTILITY PACKAGES ***
%
%\usepackage{ifpdf}
% Heiko Oberdiek's ifpdf.sty is very useful if you need conditional
% compilation based on whether the output is pdf or dvi.
% usage:
% \ifpdf
%   % pdf code
% \else
%   % dvi code
% \fi
% The latest version of ifpdf.sty can be obtained from:
% http://www.ctan.org/tex-archive/macros/latex/contrib/oberdiek/
% Also, note that IEEEtran.cls V1.7 and later provides a builtin
% \ifCLASSINFOpdf conditional that works the same way.
% When switching from latex to pdflatex and vice-versa, the compiler may
% have to be run twice to clear warning/error messages.






% *** CITATION PACKAGES ***
%
%\usepackage{cite}
% cite.sty was written by Donald Arseneau
% V1.6 and later of IEEEtran pre-defines the format of the cite.sty package
% \cite{} output to follow that of IEEE. Loading the cite package will
% result in citation numbers being automatically sorted and properly
% "compressed/ranged". e.g., [1], [9], [2], [7], [5], [6] without using
% cite.sty will become [1], [2], [5]--[7], [9] using cite.sty. cite.sty's
% \cite will automatically add leading space, if needed. Use cite.sty's
% noadjust option (cite.sty V3.8 and later) if you want to turn this off.
% cite.sty is already installed on most LaTeX systems. Be sure and use
% version 4.0 (2003-05-27) and later if using hyperref.sty. cite.sty does
% not currently provide for hyperlinked citations.
% The latest version can be obtained at:
% http://www.ctan.org/tex-archive/macros/latex/contrib/cite/
% The documentation is contained in the cite.sty file itself.






% *** GRAPHICS RELATED PACKAGES ***
%
\ifCLASSINFOpdf
  % \usepackage[pdftex]{graphicx}
  % declare the path(s) where your graphic files are
  % \graphicspath{{../pdf/}{../jpeg/}}
  % and their extensions so you won't have to specify these with
  % every instance of \includegraphics
  % \DeclareGraphicsExtensions{.pdf,.jpeg,.png}
\else
  % or other class option (dvipsone, dvipdf, if not using dvips). graphicx
  % will default to the driver specified in the system graphics.cfg if no
  % driver is specified.
  % \usepackage[dvips]{graphicx}
  % declare the path(s) where your graphic files are
  % \graphicspath{{../eps/}}
  % and their extensions so you won't have to specify these with
  % every instance of \includegraphics
  % \DeclareGraphicsExtensions{.eps}
\fi
% graphicx was written by David Carlisle and Sebastian Rahtz. It is
% required if you want graphics, photos, etc. graphicx.sty is already
% installed on most LaTeX systems. The latest version and documentation can
% be obtained at:
% http://www.ctan.org/tex-archive/macros/latex/required/graphics/
% Another good source of documentation is "Using Imported Graphics in
% LaTeX2e" by Keith Reckdahl which can be found as epslatex.ps or
% epslatex.pdf at: http://www.ctan.org/tex-archive/info/
%
% latex, and pdflatex in dvi mode, support graphics in encapsulated
% postscript (.eps) format. pdflatex in pdf mode supports graphics
% in .pdf, .jpeg, .png and .mps (metapost) formats. Users should ensure
% that all non-photo figures use a vector format (.eps, .pdf, .mps) and
% not a bitmapped formats (.jpeg, .png). IEEE frowns on bitmapped formats
% which can result in "jaggedy"/blurry rendering of lines and letters as
% well as large increases in file sizes.
%
% You can find documentation about the pdfTeX application at:
% http://www.tug.org/applications/pdftex





% *** MATH PACKAGES ***
%
%\usepackage[cmex10]{amsmath}
% A popular package from the American Mathematical Society that provides
% many useful and powerful commands for dealing with mathematics. If using
% it, be sure to load this package with the cmex10 option to ensure that
% only type 1 fonts will utilized at all point sizes. Without this option,
% it is possible that some math symbols, particularly those within
% footnotes, will be rendered in bitmap form which will result in a
% document that can not be IEEE Xplore compliant!
%
% Also, note that the amsmath package sets \interdisplaylinepenalty to 10000
% thus preventing page breaks from occurring within multiline equations. Use:
%\interdisplaylinepenalty=2500
% after loading amsmath to restore such page breaks as IEEEtran.cls normally
% does. amsmath.sty is already installed on most LaTeX systems. The latest
% version and documentation can be obtained at:
% http://www.ctan.org/tex-archive/macros/latex/required/amslatex/math/





% *** SPECIALIZED LIST PACKAGES ***
%
%\usepackage{algorithmic}
% algorithmic.sty was written by Peter Williams and Rogerio Brito.
% This package provides an algorithmic environment fo describing algorithms.
% You can use the algorithmic environment in-text or within a figure
% environment to provide for a floating algorithm. Do NOT use the algorithm
% floating environment provided by algorithm.sty (by the same authors) or
% algorithm2e.sty (by Christophe Fiorio) as IEEE does not use dedicated
% algorithm float types and packages that provide these will not provide
% correct IEEE style captions. The latest version and documentation of
% algorithmic.sty can be obtained at:
% http://www.ctan.org/tex-archive/macros/latex/contrib/algorithms/
% There is also a support site at:
% http://algorithms.berlios.de/index.html
% Also of interest may be the (relatively newer and more customizable)
% algorithmicx.sty package by Szasz Janos:
% http://www.ctan.org/tex-archive/macros/latex/contrib/algorithmicx/




% *** ALIGNMENT PACKAGES ***
%
%\usepackage{array}
% Frank Mittelbach's and David Carlisle's array.sty patches and improves
% the standard LaTeX2e array and tabular environments to provide better
% appearance and additional user controls. As the default LaTeX2e table
% generation code is lacking to the point of almost being broken with
% respect to the quality of the end results, all users are strongly
% advised to use an enhanced (at the very least that provided by array.sty)
% set of table tools. array.sty is already installed on most systems. The
% latest version and documentation can be obtained at:
% http://www.ctan.org/tex-archive/macros/latex/required/tools/


%\usepackage{mdwmath}
%\usepackage{mdwtab}
% Also highly recommended is Mark Wooding's extremely powerful MDW tools,
% especially mdwmath.sty and mdwtab.sty which are used to format equations
% and tables, respectively. The MDWtools set is already installed on most
% LaTeX systems. The lastest version and documentation is available at:
% http://www.ctan.org/tex-archive/macros/latex/contrib/mdwtools/


% IEEEtran contains the IEEEeqnarray family of commands that can be used to
% generate multiline equations as well as matrices, tables, etc., of high
% quality.


%\usepackage{eqparbox}
% Also of notable interest is Scott Pakin's eqparbox package for creating
% (automatically sized) equal width boxes - aka "natural width parboxes".
% Available at:
% http://www.ctan.org/tex-archive/macros/latex/contrib/eqparbox/





% *** SUBFIGURE PACKAGES ***
%\usepackage[tight,footnotesize]{subfigure}
% subfigure.sty was written by Steven Douglas Cochran. This package makes it
% easy to put subfigures in your figures. e.g., "Figure 1a and 1b". For IEEE
% work, it is a good idea to load it with the tight package option to reduce
% the amount of white space around the subfigures. subfigure.sty is already
% installed on most LaTeX systems. The latest version and documentation can
% be obtained at:
% http://www.ctan.org/tex-archive/obsolete/macros/latex/contrib/subfigure/
% subfigure.sty has been superceeded by subfig.sty.



%\usepackage[caption=false]{caption}
%\usepackage[font=footnotesize]{subfig}
% subfig.sty, also written by Steven Douglas Cochran, is the modern
% replacement for subfigure.sty. However, subfig.sty requires and
% automatically loads Axel Sommerfeldt's caption.sty which will override
% IEEEtran.cls handling of captions and this will result in nonIEEE style
% figure/table captions. To prevent this problem, be sure and preload
% caption.sty with its "caption=false" package option. This is will preserve
% IEEEtran.cls handing of captions. Version 1.3 (2005/06/28) and later
% (recommended due to many improvements over 1.2) of subfig.sty supports
% the caption=false option directly:
%\usepackage[caption=false,font=footnotesize]{subfig}
%
% The latest version and documentation can be obtained at:
% http://www.ctan.org/tex-archive/macros/latex/contrib/subfig/
% The latest version and documentation of caption.sty can be obtained at:
% http://www.ctan.org/tex-archive/macros/latex/contrib/caption/




% *** FLOAT PACKAGES ***
%
%\usepackage{fixltx2e}
% fixltx2e, the successor to the earlier fix2col.sty, was written by
% Frank Mittelbach and David Carlisle. This package corrects a few problems
% in the LaTeX2e kernel, the most notable of which is that in current
% LaTeX2e releases, the ordering of single and double column floats is not
% guaranteed to be preserved. Thus, an unpatched LaTeX2e can allow a
% single column figure to be placed prior to an earlier double column
% figure. The latest version and documentation can be found at:
% http://www.ctan.org/tex-archive/macros/latex/base/



%\usepackage{stfloats}
% stfloats.sty was written by Sigitas Tolusis. This package gives LaTeX2e
% the ability to do double column floats at the bottom of the page as well
% as the top. (e.g., "\begin{figure*}[!b]" is not normally possible in
% LaTeX2e). It also provides a command:
%\fnbelowfloat
% to enable the placement of footnotes below bottom floats (the standard
% LaTeX2e kernel puts them above bottom floats). This is an invasive package
% which rewrites many portions of the LaTeX2e float routines. It may not work
% with other packages that modify the LaTeX2e float routines. The latest
% version and documentation can be obtained at:
% http://www.ctan.org/tex-archive/macros/latex/contrib/sttools/
% Documentation is contained in the stfloats.sty comments as well as in the
% presfull.pdf file. Do not use the stfloats baselinefloat ability as IEEE
% does not allow \baselineskip to stretch. Authors submitting work to the
% IEEE should note that IEEE rarely uses double column equations and
% that authors should try to avoid such use. Do not be tempted to use the
% cuted.sty or midfloat.sty packages (also by Sigitas Tolusis) as IEEE does
% not format its papers in such ways.





% *** PDF, URL AND HYPERLINK PACKAGES ***
%
%\usepackage{url}
% url.sty was written by Donald Arseneau. It provides better support for
% handling and breaking URLs. url.sty is already installed on most LaTeX
% systems. The latest version can be obtained at:
% http://www.ctan.org/tex-archive/macros/latex/contrib/misc/
% Read the url.sty source comments for usage information. Basically,
% \url{my_url_here}.





% *** Do not adjust lengths that control margins, column widths, etc. ***
% *** Do not use packages that alter fonts (such as pslatex).         ***
% There should be no need to do such things with IEEEtran.cls V1.6 and later.
% (Unless specifically asked to do so by the journal or conference you plan
% to submit to, of course. )


% correct bad hyphenation here
\hyphenation{op-tical net-works semi-conduc-tor}


\usepackage{graphicx,mkolar_definitions,subfigure,algorithm,algorithmic}
\usepackage{color}
\usepackage{placeins}
\usepackage{float}
\usepackage{tabularx,colortbl}

\begin{document}
%
% paper title
% can use linebreaks \\ within to get better formatting as desired
\title{A Path Algorithm for Localizing Anomalous Activity in Graphs}


% author names and affiliations
% use a multiple column layout for up to three different
% affiliations
\author{\IEEEauthorblockN{James Sharpnack}
\IEEEauthorblockA{Machine Learning Department\\
Carnegie Mellon University\\
Pittsburgh, PA, USA\\
jsharpna@gmail.com\\}}

% conference papers do not typically use \thanks and this command
% is locked out in conference mode. If really needed, such as for
% the acknowledgment of grants, issue a \IEEEoverridecommandlockouts
% after \documentclass

% for over three affiliations, or if they all won't fit within the width
% of the page, use this alternative format:
%
%\author{\IEEEauthorblockN{Michael Shell\IEEEauthorrefmark{1},
%Homer Simpson\IEEEauthorrefmark{2},
%James Kirk\IEEEauthorrefmark{3},
%Montgomery Scott\IEEEauthorrefmark{3} and
%Eldon Tyrell\IEEEauthorrefmark{4}}
%\IEEEauthorblockA{\IEEEauthorrefmark{1}School of Electrical and Computer Engineering\\
%Georgia Institute of Technology,
%Atlanta, Georgia 30332--0250\\ Email: see http://www.michaelshell.org/contact.html}
%\IEEEauthorblockA{\IEEEauthorrefmark{2}Twentieth Century Fox, Springfield, USA\\
%Email: homer@thesimpsons.com}
%\IEEEauthorblockA{\IEEEauthorrefmark{3}Starfleet Academy, San Francisco, California 96678-2391\\
%Telephone: (800) 555--1212, Fax: (888) 555--1212}
%\IEEEauthorblockA{\IEEEauthorrefmark{4}Tyrell Inc., 123 Replicant Street, Los Angeles, California 90210--4321}}




% use for special paper notices
%\IEEEspecialpapernotice{(Invited Paper)}




% make the title area
\maketitle


\begin{abstract}
The localization of anomalous activity in graphs is a statistical problem that arises in many applications, such as network surveillance, disease outbreak detection, and activity monitoring in social networks.
We will address the localization of a cluster of activity in Gaussian noise in directed, weighted graphs.
We develop a penalized likelihood estimator (we call the relaxed graph scan) as a relaxation of the NP-hard graph scan statistic.
We review how the relaxed graph scan (RGS) can be solved using graph cuts, and outline the max-flow min-cut duality.
We use this combinatorial duality to derive a path algorithm for the RGS by solving successive max flows.
We demonstrate the effectiveness of the RGS on two simulations, over an undirected and directed graph.
\end{abstract}

\IEEEpeerreviewmaketitle

\section{Introduction}

Classically, the detection and identification of anomalies has focused on identifying rare behaviors and aberrant bursts in activity over a single data source or channel.
With the advent of large surveillance projects, social networks, and mobile computing, statistics needs to comprehensively address the detection of anomalous activity in graphs.
In reality, very little is known about the detection and localization of activity in graphs, despite a variety of real-world applications such as activity detection in social networks, network surveillance, disease outbreak detection, biomedical imaging, sensor network detection, gene network analysis, environmental monitoring and malware detection.
Recent theoretical contributions in the statistical literature\cite{arias2011detection,addario2010combinatorial} have detailed the inherent difficulty of such combinatorial statistical problems but have positive results only under restrictive conditions on the graph topology.

In machine learning and computer vision, Markov random fields (MRF) with Ising priors have been used to model activation patterns that are consistent with some graph structure.
In this Bayesian setting, the maximum a-posteriori (MAP) estimate has dominated the research, due to its computational feasibility and success in computer vision applications.
In this paper, we propose a penalized likelihood estimator that takes a similar form to the MRF MAP estimate.
We develop a path algorithm for this estimator, as the regularization parameter varies, which can be solved with successive maximum flow computations.

\subsection{Problem Setup}

Consider a connected, possibly weighted, directed graph $G$ defined by a set of vertices $V$ ($|V| = p$) and directed edges $E$ ($|E| = m$) which are ordered pairs of vertices.
We will let $u \to v$ denote an edge from the vertices $u$ to $v$.
Furthermore, the edges may be assigned weights, $\{ W(e) \}_{e \in E}$, that determine the relative strength of the interactions of the adjacent vertices.
In the graph-structured normal means problem, we observe one realization of the random vector
\begin{equation}
\label{eq:normal_means}
\yb = \xb + \xib,
\end{equation}
where $\xb \in \RR^p$, $\xib \sim N(0,\Ib_{p\times p})$.
The structure of activation pattern $\xb$ is determined by the graph $G$. 
Specifically, we assume that there is a parameter $\rho$ (possibly
dependent on $p$) such that the class of graph-structured activation patterns $\xb$ is given as follows.
\[
\Ccal = \left\{ C \subseteq V : \frac{\out(C)}{|C|} \le \rho \right\}
\]
\[
\Xcal = \left\{\xb : \xb = \mu \one_C, \mu > 0, C \in \Ccal\right\}
\]
Here $\out(C) = \sum_{u \to v\in E} W(u \to v) I\{ u \in C, v \in \bar C \}$ is the weight of edges leaving the cluster $C$.
In other words, the set of activated vertices $C$ have a small {\em cut size} in the graph $G$.
Notice that the model \eqref{eq:normal_means} is equivalent to the more general model in which $\EE \xi_i^2 = \sigma^2$ with $\sigma$ known.
As a notational convenience, if $\zb \in \RR^p$ and $C \subseteq [p]$, then we denote $\zb(C) = \sum_{v \in C} z_v$.
Throughout the study, let the edge-incidence matrix of $G$ be $\nabla \in \RR^{m \times p}$ such that for $v \to u \in E$, $\nabla_{v\to u,v} = -W(v \to u)$, $\nabla_{v \to u,u} = W(v \to u)$ and is $0$ elsewhere.

%If we wanted to consider known $\sigma^2$ then we would apply all our algorithms to $\yb / \sigma$ and replace $\mu$ with $\mu / \sigma$ in all of our statements.
%For this reason, we call $\mu$ the signal-to-noise (SNR) ratio, and proceed with $\sigma = 1$.

\subsection{Related Work}

There have been several approaches to signal processing over graphs.
Markov random fields (MRF) provide a succinct framework in which the underlying signal is modeled as a draw from an Ising or Potts model \cite{cevher2009sparse,ravikumar2006quadratic}.
A similar line of research is the use of kernels over graphs, which began with the development of diffusion kernels \cite{kondor2002diffusion}, and was extended through Green's functions on graphs \cite{smola2003kernels}.
In this study, we develop a path algorithm for a localization of structured signals, which is similar to the work of \cite{tibshirani2011solution,hoefling2010path}.

The estimation of the mean of a Gaussian has served as a canonical problem in nonparametric statistics.
When the mean is assumed to be sparse, asymptotic minimaxity has been established for thresholding and false-discovery rate based estimators \cite{donoho1995wavelet,abramovich2006adapting}.
When the mean vector is assumed to lie within an ellipsoid then Pinsker's estimator has been shown to be asymptotically optimal as $\sigma$ approaches $0$ \cite{johnstone2002function}.

In spatial statistics, it is common, when searching for anomalous activity to scan over regions in the spatial domain testing for elevated activity\cite{neill2004rapid,agarwal2006spatial}.
There have been scan statistics proposed for graphs, most notably the work of \cite{priebe2005scan} in which the authors scan over neighborhoods of the graphs defined by the graph distance.
Other work has been done on the theory and algorithms for scan statistics over specific graph models, but are not easily generalizable to arbitrary graphs \cite{yi2009unified, arias2011detection}.
More recently, it has been found that scanning over all well connected regions of a graph can be computationally intractable, and so approximations to the intractable likelihood-based procedure have been studied \cite{sharpnack2012changepoint,sharpnack2012detecting}.
We follow in this line of work, with a relaxation to the intractable restricted likelihood maximization.

\vspace{-.2cm}
\section{Method}
\label{sec:method}
\vspace{-.2cm}
As we have noted the fundamental difficulty of the hypothesis testing problem is the composite nature of the alternative hypothesis.
Because the alternative is indexed by sets, $C \in \Ccal(\rho)$, with a low cut size, it is reasonable that the test statistic that we will derive results from a combinatorial optimization program.
In fact, we will show we can express the generalized likelihood ratio (GLR) statistic in terms of a modular program with submodular constraints.
This will turn out to be a possibly NP-hard program, as a special case of such programs is the well known knapsack problem \cite{papadimitriou1998combinatorial}.
With this in mind, we provide a convex relaxation, using the Lov\'asz extension, to the ideal GLR statistic.
This relaxation conveniently has a dual objective that can be evaluated with a binary Markov random field energy minimization, which is a well understood program.
We will reserve the theoretical statistical analysis for the following section.

{\bf Submodularity.} Before we proceed, we will introduce the reader to submodularity and the Lov\'asz extension. (A very nice introduction to submodularity can be found in \cite{bach2010convex}.)
For any set, which we may as well take to be the vertex set $[p]$, we say that a function $F : \{0,1\}^p \rightarrow \RR$ is submodular if for any $A,B \subseteq [p]$, $F(A) + F(B) \ge F(A \cap B) + F(A \cup B)$. (We will interchangeably use the bijection between $2^{[p]}$ and $\{0,1\}^p$ defined by $C \to \one_C$.)
In this way, a submodular function experiences diminishing returns, as additions to large sets tend to be less dramatic than additions to small sets.
But while this diminishing returns phenomenon is akin to concave functions, for optimization purposes submodularity acts like convexity, as it admits efficient minimization procedures.
Moreover, for every submodular function there is a Lov\'asz extension $f : [0,1]^p \rightarrow \RR$ defined in the following way: for $\xb \in [0,1]^p$ let $x_{j_i}$ denote the $i$th largest element of $\xb$, then
\[
f(\xb) = x_{j_1} F(\{j_1\}) + \sum_{i=2}^{p} (F(\{ j_1,\ldots,j_i \}) - F(\{ j_1,\ldots,j_{i-1} \})) x_{j_i}
\]
Submodular functions as a class is similar to convex functions in that it is closed under addition and non-negative scalar multiplication.
The following facts about Lov\'asz extensions will be important.
\begin{proposition}{\cite{bach2010convex}}
\label{prop:submod}
Let $F$ be submodular and $f$ be its Lov\'asz extension. Then $f$ is convex, $f(\xb) = F(\xb)$ if $\xb \in \{0,1\}^p$, and 
\[
\min \{ F(\xb) : \xb \in \{0,1\}^p \} = \min \{ f(\xb) : \xb \in [0,1]^p \}
\]
\end{proposition}
We are now sufficiently prepared to develop the test statistics that will be the focus of this paper.
\vspace{-.1cm}
\subsection{Graph Scan Statistic}
\label{sec:graph_scan}
\vspace{-.1cm}
It is instructive, when faced with a class of probability distributions, indexed by subsets $\Ccal \subseteq 2^{[p]}$, to think about what techniques we would use if we knew the correct set $C \in \Ccal$ (which is often called oracle information).
One would in this case be only testing the null hypothesis $H_0: \xb = \zero$ against the simple alternative $H_1: \xb \propto \one_C$.
In this situation, we would employ the likelihood ratio test because by the Neyman-Pearson lemma it is the uniformly most powerful test statistic.
The maximum likelihood estimator for $\xb$ is $\one_C \one_C^\top \yb / |C|$ (the MLE of $\mu$ is $\one_C^\top \yb /\sqrt{|C|}$) and the likelihood ratio turns out to be 
\[
\exp \left\{ - \frac 12 \| \yb \|^2 \right\} / \exp \left\{ - \frac 12 \left\| \frac{\one_C \one_C^\top \yb}{|C|} - \yb \right\|^2 \right\} = \exp\left\{ \frac{(\one_C^\top \yb)^2}{2|C|} \right\}
\]
Hence, the log-likelihood ratio is proportional to $(\one_C^\top \yb)^2/|C|$ and thresholding this at $z^2_{1 - \alpha/2}$ gives us a size $\alpha$ test.

This reasoning has been subject to the assumption that we had oracle knowledge of $C$.
A natural statistic, when $C$ is unknown, is the generalized log-likelihood ratio (GLR) defined by $\max (\one_C^\top \yb)^2/|C| \textrm{ s.t. } C \in \Ccal$.
We will work with the {\em graph scan statistic} (GSS),
\begin{equation}
\label{eq:gss}
\hat s = \max \frac{\one_C^\top \yb}{\sqrt{|C|}} \textrm{ s.t. } C \in \Ccal(\rho) = \{ C : \out(C) \le \rho \}
\end{equation}
which is nearly equivalent to the GLR. (We can in fact evaluate $\hat s$ for $\yb$ and $-\yb$, taking a maximum and obtain the GLR, but statistically this is nearly the same.)
Notice that there is no guarantee that the program above is computationally feasible.
In fact, it belongs to a class of programs, specifically modular programs with submodular constraints that is known to contain NP-hard instantiations, such as the ratio cut program and the knapsack program \cite{papadimitriou1998combinatorial}.
Hence, we are compelled to form a relaxation of the above program, that will with luck provide a feasible algorithm.
\vspace{-.1cm}
\subsection{Lov\'asz Extended Scan Statistic}
\label{sec:less}
\vspace{-.1cm}
It is common, when faced with combinatorial optimization programs that are computationally infeasible, to relax the domain from the discrete $\{0,1\}^p$ to a continuous domain, such as $[0,1]^p$.
Generally, the hope is that optimizing the relaxation will approximate the combinatorial program well.
First we require that we can relax the constraint $\out(C) \le \rho$ to the hypercube $[0,1]^p$.
This will be accomplished by replacing it with its Lov\'asz extension $\|(\nabla \xb)_+ \|_1 \le \rho$.
We then form the relaxed program, which we will call the {\em Lov\'asz extended scan statistic} (LESS), 
\begin{equation}
\label{eq:less}
\hat l = \max_{t \in [p]} \max_\xb \frac{\xb^\top \yb}{\sqrt{t}} \textrm{ s.t. } \xb \in \Xcal(\rho,t) = \{ \xb \in [0,1]^p : \| (\nabla \xb)_+ \|_1 \le \rho, \one^\top \xb \le t \}
\end{equation}
We will find that not only can this be solved with a convex program, but the dual objective is a minimum binary Markov random field energy program.
To this end, we will briefly go over binary Markov random fields, which we will find can be used to solve our relaxation.

{\bf Binary Markov Random Fields.} Much of the previous work on graph structured statistical procedures assumes a Markov random field (MRF) model, in which there are discrete labels assigned to each vertex in $[p]$, and the observed variables $\{y_v\}_{v \in [p]}$ are conditionally independent given these labels.
Furthermore, the prior distribution on the labels is drawn according to an Ising model (if the labels are binary) or a Potts model otherwise. 
The task is to then compute a Bayes rule from the posterior of the MRF.
The majority of the previous work assumes that we are interested in the maximum a-posteriori (MAP) estimator, which is the Bayes rule for the $0/1$-loss.
This can generally be written in the form,
\[
\min_{\xb \in \{0,1\}^p} \sum_{v \in [p]} -l_v(x_v | y_v) + \sum_{v \ne u \in [p]} W_{v,u} I\{ x_v \ne x_u\}
\]
where $l_v$ is a data dependent log-likelihood.
Such programs are called graph-representable in \cite{kolmogorov2004energy}, and are known to be solvable in the binary case with $s$-$t$ graph cuts.
Thus, by the min-cut max-flow theorem the value of the MAP objective can be obtained by computing a maximum flow.
More recently, a dual-decomposition algorithm has been developed in order to parallelize the computation of the MAP estimator for binary MRFs \cite{strandmark2010parallel,sontag2011introduction}.

We are now ready to state our result regarding the dual form of the LESS program, \eqref{eq:less}.
\begin{proposition}
\label{prop:less_alg}
Let $\eta_0, \eta_1 \ge 0$, and define the dual function of the LESS,
\[
g(\eta_0,\eta_1) = \max_{\xb \in \{0,1\}^p} \yb^\top \xb - \eta_0 \one^\top \xb - \eta_1 \| \nabla \xb \|_0
\]
The LESS estimator is equal to the following minimum of convex optimizations
\[
\hat l = \max_{t \in [p]} \frac{1}{\sqrt{t}} \min_{\eta_0,\eta_1 \ge 0} g(\eta_0,\eta_1) + \eta_0 t + \eta_1 \rho
\]
$g(\eta_0,\eta_1)$ is the objective of a MRF MAP problem, which is poly-time solvable with $s$-$t$ graph cuts.
\end{proposition}

\section{Experiments}

%We have provided a path algorithm for our statistic, $\hat C$, which was motivated from first principles.
We will now conclude with an empirical study of the effectiveness of Algorithm \ref{alg:flowpath} on some simulations.
Notice that a specific max flow algorithm was not prescribed when computing the gradient flow.
In our implementation, we use the Edmonds-Karp algorithm, in which residual flows are found using breadth-first search.
More recently, a dual-decomposition algorithm has been developed in order to parallelize the computation of the MAP estimator for binary MRFs \cite{strandmark2010parallel,sontag2011introduction}.

\begin{figure*}
\centering{
\includegraphics[width=4cm]{rect_n=15_true.eps}\hspace{1cm}
\includegraphics[width=4cm]{rect_n=15_noisy.eps}\hspace{1cm}
\includegraphics[width=4cm]{rect_n=15_hat.eps}}
\caption{The undirected $25 \times 25$ lattice graph was used with a $4 \times 4$ rectangular region $C^\star$ (left).  The signal size used was $\mu = 2.35$ with $\sigma = 1$ ($\yb$ is depicted in the middle).  The best reconstruction is displayed (right).}
\label{fig1}
%\vspace{-.25cm}
\end{figure*}


\begin{figure*}
\centering{
\includegraphics[width=4cm]{dir_n=15_true.eps}\hspace{1cm}
\includegraphics[width=4cm]{dir_n=15_noisy.eps}\hspace{1cm}
\includegraphics[width=4cm]{dir_n=15_hat.eps}}
\caption{A directed graph (depicted by the arrows) was constructed with edges between all $8$ neighboring blocks in which more weight is on edges that descend to the right.  A region was chosen that has a low cut size (left).  The signal size is $\mu = 1.75$ and $\sigma = 1$ ($\yb$ is depicted in the middle).  The best reconstruction is displayed (right).}
\label{fig2}
\vspace{-.25cm}
\end{figure*}

\begin{figure*}
\centering{
\includegraphics[width=5.5cm]{rect_n=15_MSE_path.eps}
\includegraphics[width=5.5cm]{dir_n=15_MSE_path.eps}}
\caption{MSE as a function of regularization parameter $\nu$.  The lattice graph example is left and the directed graph example is right.}
\label{fig3}
\vspace{-.25cm}
\end{figure*}

We construct an undirected, unweighted lattice graph by identifying each vertex with a square in a $15 \times 15$ grid, and adjoining vertices that share an side of the square with weight $1$.
A $4 \times 4$ rectangle was constructed to be $C^\star$ and the signal size $\mu = 2.35$ with noise level $\sigma = 1$. (Figure \ref{fig1} depicts the cluster, noisy observations and reconstruction.)
The smallest MSE (Hamming distance between $C^\star$ and $\hat C$) in the regularization path was $2$. (The MSE throughout the regularization path is given in Figure \ref{fig3} (left)).

We form a weighted directed graph by associating the vertices to squares in a grid ($\{(i,j) : i, j \in [15]\}$) such that $(i_1,j_1), (i_2,j_2)$ have an edge between them if $|i_1 - i_2| \le 1$ and $|j_1 - j_2| \le 1$.  
Moreover, the edge weight is equal to $e^{(i_2 - i_1 + j_2 - i_1)/3}$ so that the weight is larger in the direction of the arrows of Figure \ref{fig2}.
$C^\star$ contains $16$ vertices depicted in Figure \ref{fig2} (left), the signal size is $\mu = 1.75$ and $\sigma = 1$.
Again the MSE is given in Figure \ref{fig3} (right).
It has been demonstrated that the RGS can successfully reconstruct the true cluster of activation $C^\star$.
This can also be done efficiently with the Flow Path algorithm, which gives us the entire regularization path for the RGS.
The algorithm above is computationally slower than the spanning tree wavelet \cite{sharpnack2012detecting}, but it experimentally dominates it and other pre-existing methods \cite{sharpnack2013near-optimal}.
%The graph scan statistic and the regularized graph scan are novel because they are able to scan over a combinatorial class of clusters defined using graph cuts.

%\section{Discussion}


%\subsection{References}
%\section*{ACKNOWLEDGEMENT}
%\begin{thebibliography}{1}
%\end{thebibliography}

\bibliographystyle{unsrt}
\bibliography{biblio}

\end{document}




