\section{Introduction}

Classically, the detection and identification of anomalies has focused on identifying rare behaviors and aberrant bursts in activity over a single data source or channel.
With the advent of large surveillance projects, social networks, and mobile computing, statistics needs to comprehensively address the detection of anomalous activity in graphs.
In reality, very little is known about the detection and localization of activity in graphs, despite a variety of real-world applications such as activity detection in social networks, network surveillance, disease outbreak detection, biomedical imaging, sensor network detection, gene network analysis, environmental monitoring and malware detection.
Recent theoretical contributions in the statistical literature\cite{arias2011detection,addario2010combinatorial} have detailed the inherent difficulty of such combinatorial statistical problems but have positive results only under restrictive conditions on the graph topology.

In machine learning and computer vision, Markov random fields (MRF) with Ising priors have been used to model activation patterns that are consistent with some graph structure.
In this Bayesian setting, the maximum a-posteriori (MAP) estimate has dominated the research, due to its computational feasibility and success in computer vision applications.
In this paper, we propose a penalized likelihood estimator that takes a similar form to the MRF MAP estimate.
We develop a path algorithm for this estimator, as the regularization parameter varies, which can be solved with successive maximum flow computations.

\subsection{Problem Setup}

Consider a connected, possibly weighted, directed graph $G$ defined by a set of vertices $V$ ($|V| = p$) and directed edges $E$ ($|E| = m$) which are ordered pairs of vertices.
We will let $u \to v$ denote an edge from the vertices $u$ to $v$.
Furthermore, the edges may be assigned weights, $\{ W(e) \}_{e \in E}$, that determine the relative strength of the interactions of the adjacent vertices.
In the graph-structured normal means problem, we observe one realization of the random vector
\begin{equation}
\label{eq:normal_means}
\yb = \xb + \xib,
\end{equation}
where $\xb \in \RR^p$, $\xib \sim N(0,\Ib_{p\times p})$.
The structure of activation pattern $\xb$ is determined by the graph $G$. 
Specifically, we assume that there is a parameter $\rho$ (possibly
dependent on $p$) such that the class of graph-structured activation patterns $\xb$ is given as follows.
\[
\Ccal = \left\{ C \subseteq V : \frac{\out(C)}{|C|} \le \rho \right\}
\]
\[
\Xcal = \left\{\xb : \xb = \mu \one_C, \mu > 0, C \in \Ccal\right\}
\]
Here $\out(C) = \sum_{u \to v\in E} W(u \to v) I\{ u \in C, v \in \bar C \}$ is the weight of edges leaving the cluster $C$.
In other words, the set of activated vertices $C$ have a small {\em cut size} in the graph $G$.
Notice that the model \eqref{eq:normal_means} is equivalent to the more general model in which $\EE \xi_i^2 = \sigma^2$ with $\sigma$ known.
As a notational convenience, if $\zb \in \RR^p$ and $C \subseteq [p]$, then we denote $\zb(C) = \sum_{v \in C} z_v$.
Throughout the study, let the edge-incidence matrix of $G$ be $\nabla \in \RR^{m \times p}$ such that for $v \to u \in E$, $\nabla_{v\to u,v} = -W(v \to u)$, $\nabla_{v \to u,u} = W(v \to u)$ and is $0$ elsewhere.

%If we wanted to consider known $\sigma^2$ then we would apply all our algorithms to $\yb / \sigma$ and replace $\mu$ with $\mu / \sigma$ in all of our statements.
%For this reason, we call $\mu$ the signal-to-noise (SNR) ratio, and proceed with $\sigma = 1$.

\subsection{Related Work}

There have been several approaches to signal processing over graphs.
Markov random fields (MRF) provide a succinct framework in which the underlying signal is modeled as a draw from an Ising or Potts model \cite{cevher2009sparse,ravikumar2006quadratic}.
A similar line of research is the use of kernels over graphs, which began with the development of diffusion kernels \cite{kondor2002diffusion}, and was extended through Green's functions on graphs \cite{smola2003kernels}.
In this study, we develop a path algorithm for a localization of structured signals, which is similar to the work of \cite{tibshirani2011solution,hoefling2010path}.

The estimation of the mean of a Gaussian has served as a canonical problem in nonparametric statistics.
When the mean is assumed to be sparse, asymptotic minimaxity has been established for thresholding and false-discovery rate based estimators \cite{donoho1995wavelet,abramovich2006adapting}.
When the mean vector is assumed to lie within an ellipsoid then Pinsker's estimator has been shown to be asymptotically optimal as $\sigma$ approaches $0$ \cite{johnstone2002function}.

In spatial statistics, it is common, when searching for anomalous activity to scan over regions in the spatial domain testing for elevated activity\cite{neill2004rapid,agarwal2006spatial}.
There have been scan statistics proposed for graphs, most notably the work of \cite{priebe2005scan} in which the authors scan over neighborhoods of the graphs defined by the graph distance.
Other work has been done on the theory and algorithms for scan statistics over specific graph models, but are not easily generalizable to arbitrary graphs \cite{yi2009unified, arias2011detection}.
More recently, it has been found that scanning over all well connected regions of a graph can be computationally intractable, and so approximations to the intractable likelihood-based procedure have been studied \cite{sharpnack2012changepoint,sharpnack2012detecting}.
We follow in this line of work, with a relaxation to the intractable restricted likelihood maximization.
