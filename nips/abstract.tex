\begin{abstract}

The detection of anomalous activity in graphs is a statistical problem that arises in many applications, such as network surveillance, disease outbreak detection, and activity monitoring in social networks. 
Beyond its wide applicability, graph structured anomaly detection serves as a case study in the difficulty of balancing computational complexity with statistical power.
In this work, we develop from first principles the generalized likelihood ratio test for determining if there is a well connected region of activation over the vertices in the graph in Gaussian noise.
Because this test is computationally infeasible, we provide a relaxation, called the Lov\'asz extended scan statistic (LESS) that uses submodularity to approximate the intractable generalized likelihood ratio.
We demonstrate a connection between LESS and maximum a-posteriori inference in Markov random fields, which provides us with a poly-time algorithm for LESS.
Using electrical network theory, we are able to control type 1 error for LESS and prove conditions under which LESS is risk consistent.
Finally, we consider specific graph models, the torus, $k$-nearest neighbor graphs, and $\epsilon$-random graphs.
We show that on these graphs our results provide near-optimal performance by matching our results to known lower bounds.

\end{abstract}


