\vspace{-.2cm}
\section{Related Work}
\label{sec:related}
\vspace{-.2cm}
{\bf Graph structured signal processing.}
There have been several approaches to signal processing over graphs.
Markov random fields (MRF) provide a succinct framework in which the underlying signal is modeled as a draw from an Ising or Potts model \cite{cevher2009sparse,ravikumar2006quadratic}.
We will return to MRFs in a later section, as it will relate to our scan statistic.
A similar line of research is the use of kernels over graphs.
The study of kernels over graphs began with the development of diffusion kernels \cite{kondor2002diffusion}, and was extended through Green's functions on graphs \cite{smola2003kernels}.
%A related body of work extends marginalized kernels to graphs \cite{kashima2003marginalized,mahe2004extensions}, while recently it has been shown that this and the aforementioned definitions are members of an overarching framework with computationally efficient constructions \cite{vishwanathan2008graph}.
%Unfortunately, all of these procedures, while related, do not apply directly to the detection of anomalous activity with Gaussian noise.
While these methods are used to estimate binary signals (where $x_i \in \{0,1\}$) over graphs, little is known about their statistical properties and their use in signal detection.
To the best of our knowledge, this paper is the first connection made between anomaly detection and MRFs.

{\bf Normal means testing.}
Normal means testing in high-dimensions is a well established and fundamental problem in statistics. 
Much is known when $H_1$ derives from a smooth function space such as Besov spaces or Sobolev spaces\cite{ingster1987minimax,ingster2003nonparametric}.
Only recently have combinatorial structures such as graphs been proposed as the underlying structure of $H_1$.
A significant portion of the recent work in this area \cite{arias2005nearoptimal, arias2008searching, arias2011detection,addario2010combinatorial} has focused on incorporating structural assumptions on the signal, as a way to mitigate the effect of high-dimensionality and also because many real-life problems can be represented as instances of the normal means problem with graph-structured signals (see, for an example, \cite{jacob2010gains}).

{\bf Graph scan statistics.}
In spatial statistics, it is common, when searching for anomalous activity to scan over regions in the spatial domain, testing for elevated activity\cite{neill2004rapid,agarwal2006spatial}.
There have been scan statistics proposed for graphs, most notably the work of \cite{priebe2005scan} in which the authors scan over neighborhoods of the graphs defined by the graph distance.
Other work has been done on the theory and algorithms for scan statistics over specific graph models, but are not easily generalizable to arbitrary graphs \cite{yi2009unified, arias2011detection}.
More recently, it has been found that scanning over all well connected regions of a graph can be computationally intractable, and so approximations to the intractable likelihood-based procedure have been studied \cite{sharpnack2012changepoint,sharpnack2012detecting}.
We follow in this line of work, with a relaxation to the intractable generalized likelihood ratio test.

%{\bf Graph-structured signal recovery.}

%{\bf Graph-structured detection.}

